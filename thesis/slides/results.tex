\section{Untersuchung des Verfahrens}

\subsection*{\texorpdfstring{Abschätzung von $k_\mathrm{B}$}{Abschätzung von kB}}
\begin{frame}
  \frametitle{Abschätzung von $k_\mathrm{B}$}
  \begin{itemize}
    \setlength{\itemsep}{0.3cm}

    \item zwei Primzahlen $A=104723$ und $B=66889$ gewählt
    \item Semiprimzahl $N=A\cdot B=7004816747$ mit $n=33$
    \pause{}
    \item verschiedene Werte $1^3\leq k_\mathrm{B}\leq 16^3$
    \item für jeden Wert $4800$ Wiederhohlungen
    \pause{}
    \item $N_a=1000$, $N_c=80000$, $T_0=1$, $F_c=0.997$
    \item Messung von Laufzeit und Erfolgsrate
  \end{itemize}
\end{frame}
\begin{frame}
  \frametitle{Erfolgsrate}
  \begin{figure}[H]
    \centering
    \includegraphics[width=\textwidth,height=0.8\textheight,keepaspectratio]{plot/kbguess/success.pdf}
  \end{figure}
\end{frame}
\begin{frame}
  \frametitle{Laufzeit}
  \begin{figure}[H]
    \centering
    \includegraphics[width=\textwidth,height=0.8\textheight,keepaspectratio]{plot/kbguess/time.pdf}
  \end{figure}
\end{frame}

\subsection*{Einzelner Zerlegungsschritt}
\begin{frame}
  \frametitle{Einzelner Zerlegungsschritt}
  \begin{itemize}
    \setlength{\itemsep}{0.3cm}

    \item zufällige Semiprimzahlen/allgemeine Zahlen
    \pause{}
    \item $N_a=500$, $N_c=1000$, $T_0=1$, $F_c=0.997$
    \item $k_\mathrm{B}$ automatisch abgeschätzt
    \pause{}
    \item jede Zahl mehrfach zerlegen
  \end{itemize}
\end{frame}
\begin{frame}
  \frametitle{Semiprimzahlen: Erfolgsrate}
  \begin{figure}[H]
    \centering
    \includegraphics[width=\textwidth,height=0.8\textheight,keepaspectratio]{plot/runtime/success.pdf}
  \end{figure}
\end{frame}
\begin{frame}
  \frametitle{Semiprimzahlen: Laufzeit}
  \begin{figure}[H]
    \centering
    \includegraphics[width=\textwidth,height=0.8\textheight,keepaspectratio]{plot/runtime/time.pdf}
  \end{figure}
\end{frame}
\begin{frame}
  \frametitle{Allgemeine Zahlen}
  \begin{figure}[H]
    \centering
    \includegraphics[width=\textwidth,height=0.8\textheight,keepaspectratio]{plot/runtime2/time.pdf}
  \end{figure}
\end{frame}

\subsection*{Komplette Faktorisierung}
\begin{frame}
  \frametitle{Komplette Faktorisierung}
  \begin{figure}[H]
    \centering
    \includegraphics[width=\textwidth,height=0.8\textheight,keepaspectratio]{plot/factorization/time.pdf}
  \end{figure}
\end{frame}

\subsection*{Parallelisierbarkeit}
\begin{frame}
  \frametitle{Parallelisierbarkeit}
  \begin{figure}[H]
    \centering
    \includegraphics[width=\textwidth,height=0.8\textheight,keepaspectratio]{plot/parallelization/plot.pdf}
  \end{figure}
\end{frame}
